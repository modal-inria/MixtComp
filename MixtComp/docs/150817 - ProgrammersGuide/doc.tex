\documentclass{article}

\usepackage[utf8]{inputenc}
\usepackage[english]{babel}

\usepackage[margin=1in]{geometry}
\usepackage{amsmath}
\usepackage{amssymb}
\usepackage{graphicx}

\title{MixtComp Programmer Guide}
\author{Vincent KUBICKI}
\date{17th August 2015}

\begin{document}

\maketitle
\tableofcontents

\begin{abstract}

The MixtComp code is documented inline, but it might be difficult at times to get the big picture. Some sequences, like the initialization of the computations, might be complex and vary from one model to another. Some models use straight forward analytical estimator, while some use Markov chains that need to be initialized. The architecture must accomodate all kind of situations while keeping complexity to a bearable level when implementing a new model. Hence there are key steps in the initializations, and a new model must dispatch its various steps using those provided steps.

\end{abstract}

\section{Environment variables}

The following environment variables are used:

\begin{itemize}
	\item MC{\_}VERBOSE: change the level of details output to the standard output.
	\item MC{\_}DEBUG, MC{\_}DEBUG{\_}NEW:  some debug flags.
	\item MC{\_}DETERMINISTIC: change the seeding value, from pseudo-random to 0. Used to run the unit test in a deterministic way.
\end{itemize}

\section{Global algorithm}

\section{Sequence of initialization}

There are several objects that are key to the initialization of MixtComp. Some of those objects are template argument to templated classes. For example, a MixtureManager object needs a DataHandler type argument. But the DataHandler is an interface to read and transform data, so there must be different DataHandler to cover the various inputs. The first DataHandler that has been written was DataHandlerR, which transforms the data structure obtained from R in a data structure suitable for MixtComp.

The MixtureManager reads the list of models and instantiate the corresponding C++ objects (from classes that all inherit from IMixture). It then register it to the MixtureComposer which will take care of it from there to the end of the run.

MixtureComposer::setDataParam resize various arrays in the MixtureComposer, and then calls various IMixture setDataparam, which use the DataHandler they got at construction time to get the correct data, to get the parameters from the ParamSetter in prediction mode, and to resize all important arrays.

Note that setDataParam is not supposed to initialize computations. Because MixtComp might have to restart a computation (if a degeneracy is detected for example), and setDataParam is only called once at the beginning. The initialization at the start of each computation is instead managed by MixtureComposer::InitParam, which is called in the Strategy, and not during the global initialization.

initData and initParam are both called at the beginning of SemStrategy::initSEMCheck and SemStrategy::initSEMNoCheck.

The operations performed by all variables are:

\begin{itemize}
\item getting the data from the DataHandler
\item checking the data validity (its range for example)
\item initializing various storages, including the storages for the statistics
\end{itemize}

The specific operations needed by the various models are described in the table \ref{tab:init}.

\begin{table}
	
\centering

\begin{tabular}{|c|c|c|c|}
	\hline 
	&  setDataParam & initData & initParam \\ 
	\hline 
	SimpleMixtureBridge &  Nothing special. &  AugmentedData::removeMissing & Nothing. \\ 
	\hline 
	Functional & Parse data and parameters compute quantiles. &  Function::removeMissingQuantile & Nothing. \\ 
	\hline 
	Ordinal &  setPath set the head and tail of each path &  BOSPath::initPath & sampleMuFreq, then initBOS (GibbsSampling). \\ 
	\hline 
	Rank & Parse and check data. &  RankIndividual::removeMissing & mu initialized by direct sampling, no Gibbs performed. \\ 
	\hline 
\end{tabular} 

\caption{Operations performed in the various steps of initialization, for each model}

\label{tab:init}

\end{table}

\section{Questions}

\begin{itemize}

\item Why is there a Gibbs in initParam for Ordinal, but not for Rank ? Because there are possible 0 probability completed individuals for Ordinal, but not for Rank ?
	
\end{itemize}

\end{document}
